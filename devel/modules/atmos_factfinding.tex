% Copyright (c) 2012-2018 by the GalSim developers team on GitHub
% https://github.com/GalSim-developers
%
% This file is part of GalSim: The modular galaxy image simulation toolkit.
% https://github.com/GalSim-developers/GalSim
%
% GalSim is free software: redistribution and use in source and binary forms,
% with or without modification, are permitted provided that the following
% conditions are met:
%
% 1. Redistributions of source code must retain the above copyright notice, this
%    list of conditions, and the disclaimer given in the accompanying LICENSE
%    file.
% 2. Redistributions in binary form must reproduce the above copyright notice,
%    this list of conditions, and the disclaimer given in the documentation
%    and/or other materials provided with the distribution.
%

\documentclass[preprint]{aastex}

% packages for figures
\usepackage{graphicx}
% packages for symbols
\usepackage{latexsym,amssymb}
% AMS-LaTeX package for e.g. subequations
\usepackage{amsmath}

%=====================================================================
% FRONT MATTER
%=====================================================================

\slugcomment{Draft \today}

%=====================================================================
% BEGIN DOCUMENT
%=====================================================================

\newcommand{\arroyo}{{\sc Arroyo}}
\newcommand{\laopack}{{\sc LAOPack}}
\newcommand{\imsim}{{\sc ImSim}}
\newcommand{\galsim}{{\sc GalSim}}

\begin{document}

\title{Fact-finding about turbulence codes for the atmosphere module (Issue \#169)}

\begin{abstract}
This document describes the results of a fact-finding exercise undertaken by J\"{o}rg Dietrich and 
Barney Rowe for adding the stochastic modelling of PSFs due to turbulent atmospheres to 
\galsim.  This investigation included making a preliminary judgement as to what might be used or 
borrowed from existing codes that are publicly available and suitably licensed.
\end{abstract}

\section{Motivation}
The Code Plan document drawn up with the help of the full GREAT3
collaboration stated one of the basic requirements for the software to be:
\begin{enumerate}
\item[] \emph{2. The ability to generate realistic PSF models, by which we mean
  a physically-motivated model combining both an atmospheric component
  and a specifiable telescope component, as well as simple
  parameterized PSF models.}\footnote{Full plan document available on
  the Wiki page: \newline 
http://great3.pbworks.com/w/page/49339085/Plan\%20documents\%20for\%20the\%20GREAT3\%20Code\%20and\%20Challenge}
\end{enumerate}
One of the primary initial goals of the \galsim\ project is addressing the
software needs of GREAT3, so there is interest in tackling this
problem from the lensing community.  The code is already capable of
generating PSFs due to aberrated telescope optics and combining these
with physically well-motivated approximations to an ensemble average atmospheric seeing
kernel, broadly equivalent to long duration
exposures.  These models take the form of the Double Gaussian and
Kolmogorov PSFs currently implemented in \galsim.

However, the code cannot currently generate a PSF that encompasses a
model of \emph{instantaneous} atmospheric turbulence in order to
create a more
realistic, exposure time-dependent PSF model.  This is a regime of
interest when modelling short-exposure survey imaging from the ground,
as is being seriously contemplated by a several upcoming surveys.
These PSFs may look instantaneously much `better' or `worse' than
the averaged pattern, and the impact this variability has on survey
data analysis is not yet fully quantified.

\section{Simplified physical model}
\subsection{Phase screens}\label{sect:phasescreens}
Photons passing through the atmosphere encounter a complex, dynamic
system of variable refractive index in three dimensions.  An
approximate model that has given good results in simulations of adaptive
optics is to consider the propagation of a plane wavefront through
multiple, discrete, planar \emph{phase screens} to mimic the dominant phase
distortion effects of the atmosphere. This model is seemingly used
in most codes, although see also Section \ref{sect:lsst}. 

These phase screens are
typically modelled as independent realizations of a random field with
a theoretically-motivated power spectrum, such as the von K\`{a}rm\`{a}n:
\begin{equation}\label{eq:ps}
P_{\phi}(\nu) = 0.023 r_0^2 \left(\nu^2 + 1/L_0^2\right)^{-11/6},
\end{equation}
where $r_0$ is the wavelength-dependent \emph{Fried parameter} in metres, $\nu$
is a 2D spatial frequency in the plane of the phase screen in units of cycles metre$^{-1}$, and $L_0$ is
the \emph{outer scale} of turbulence in metres.  Physically, $L_0$
corresponds to the largest size of turbulent cells in the atmosphere,
the scale on which progenitor eddies are created, measured to be $L_0
\sim$ 1-100m depending on altitude and atmospheric conditions.  For
$\nu \gg 1/L_0$, the scale-free Kolmogorov turbulence spectrum is recovered.  As
expressed, the units of $P_{\phi}(\nu)$ are (radians of
wavefront phase)$^2$/(cycles metre$^{-1}$)$^2$.

Generating images of phase screen realizations $\phi(x, y)$ with a power spectrum such as
$P_{\phi}(\nu)$ is relatively straightforward using the discrete
Fourier transform (DFT):
\begin{itemize}
\item Generate an image of a circular complex Gaussian random field with real and imaginary
parts having unit variance and zero mean, and associate $[i, j]$ pixel
locations with spatial frequencies $[\nu_x, \nu_y]$.
\item Multiply this image by the square
root of the power spectrum corresponding to $\nu^2 = {\nu_x^2 + \nu_y^2}$ at each location.
\item The DFT of the result provides in its
real and imaginary parts two independent images of phase
screens $\phi(x, y)$ with the appropriate statistical properties.  
\end{itemize}

Some care must be taken in ensuring that
all scales are sufficiently well-represented in this model. The
smallest physical scales (the phase screen image pixel scale) should be chosen to
correspond to the Nyquist frequency for the eventual
angular bandlimit of the telescope PSF, meaning this choice may
therefore also be dependent on the phase screen altitude.
The largest scales (the image extent in metres) should be large
compared to $L_0$ to ensure that the lowest frequency modes are not
heavily folded.  Mitigating strategies for efficiently overcoming this
latter requirement are discussed in Lane, Glindemann \& Dainty (1992),
but it is not clear whether this will be an issue given advances in
computing resources over the last two decades.  However, phase screen
motion due to wind (see Section \ref{sect:wind}) also increases the required
image extent.


\subsection{Propagation between multiple phase screens}
An approximate model of the phase distortion effects of a finite depth
atmosphere can then be constructed by combining multiple phase screens at different
altitudes.  The simplest approximation to this process is simply to
take a weighted average of multiple phase screens.

A more accurate approach is to model the propagation of light between
screens using the Fresnel approximation to the Huygens' integral for a
complex wavefront propagating paraxially from a plane at $z=0$ to a
plane at $z=H$.  If an infinite uniform plane wavefront has a field
distribution $\mathcal{E}(x, y, z=0)$, then its Fourier
transform in the plane at $z=H$ is given by
\begin{equation}\label{eq:fresnel}
\tilde{\mathcal{E}}(\nu_x, \nu_y, H) = {\rm e}^{-{\rm i} \pi \nu^2  H
  \lambda} \times \tilde{\mathcal{E}}(\nu_x, \nu_y, 0),
\end{equation}
omitting the wavefront amplitude distortions in this approximate
treatment (***BARNEY TODO: Must check this calculation thoroughly and
will add refs.***).   Considering phase only, we will
set $\mathcal{E}(x, y, 0) = {\rm e}^{{\rm i}\phi(x, y, 0)}$ and
$\mathcal{E}(x, y, H) = {\rm e}^{{\rm i}\phi(x, y, H)}$ when using
equation \eqref{eq:fresnel} to propagate distortions from one
screen to the next.
For separations $H$ between phase screens
of order kilometres at optical wavelengths, it can be seen that the modifications of the phase pattern
caused by Fresnel diffraction will be small, particularly for the
lowest $\nu \sim 1/L_0$ modes which contain the greatest power in $P_{\phi}(\nu)$.

To propagate through multiple phase screens using the Fresnel
approximation will therefore multiple DFT calculations at each layer,
one for generating the initial screens as described in the previous
Section, and another two (forward and inverse) for propagating light
according to equation \eqref{eq:fresnel}.  It remains to be tested
whether the Fresnel diffraction step can be safely omitted to speed
things up while retaining sufficient accuracy.

It should finally be noted that the atmospheric PSF incident across
the telescope field of view needs to be given in angular units, requiring
appropriate conversion for each phase screen and further approximation if using Fresnel diffraction.

\subsection{Screen motion due to wind}\label{sect:wind}
To simulate time variation in the patterns, phase screens can be
`rolled' using assumed wind velocities for each layer of the
atmosphere, summing to generate an exposure time-dependent average pattern.  The time step between
these summations depends on the wind velocity and the pixel scale of the phase screens, which in
turn is governed by the resolution of the telescope, as discussed in
Section \ref{sect:phasescreens}. Time
steps should be short enough that the pixel shift between summations is $\sim 0.5$\,pixels.
As the power spectrum $P_{\phi}(\nu)$ due to turbulence
is isotropic in general it may be possible to approximate independent
wind velocities at multiple layers using rectangular strips of phase screen
realizations rather than large square images.  This may be a useful
computational saving as the need to roll phase screens might add
significantly to their extent.



\subsection{A non-DFT approach?}\label{sect:lsst}
It is tantalisingly hinted in documents describing the Large Synoptic
Survey Telescope (LSST) \imsim\ data
simulator\footnote{http://lsst.astro.washington.edu/docs/overview/}
that these authors use a different method:
\begin{quote}
\emph{``The single
photon history is traced through each layer of the
atmosphere via a newly invented technique that
avoids the need to do Fourier transforms of the
wavefront perturbations. The approach is only
valid for large aperture telescopes with exposure
times of at least 10-20 seconds.''}
\end{quote}
It would, of course, be very interesting to learn more about this
technique if it is available in the public domain.

\section{Some existing codes}
The adaptive optics community has created a large variety of codes, most of which are private and
were never intended for public release. Two codes that are publicly available are

\begin{itemize}

\item \arroyo\ -- http://cfao.ucolick.org/software/arroyo.php

  A C++ class library for time domain studies of electromagnetic waves. A very large and exhaustive
  library, which goes well beyond what we believe we need in \galsim.


\item \laopack\ --
  http://lao.ucolick.org/data/Pyramid/MTF\%20Data/Code/LAOPack/LAO/pro/...
  simulationAnalysisTools.html

  A collection of IDL routines for adaptive optics, including the generation of Kolmogorov/von
  K\`arm\`an phase screens and the propagation of photons.
  
\end{itemize}

Also known to exist is

\begin{itemize}

\item \imsim\ -- http://lsst.astro.washington.edu/

  The image simulation software for the LSST, which apparently implements an entirely new method
  mentioned above.

\end{itemize}



\section{Preliminary conclusions}
None of the existing codes explored appeared to represent an ideal fit
for \galsim.  They were found to add a significant extra dependency (e.g.\
\arroyo, \imsim) and are perhaps over-powered (\arroyo) or
specifically tailored to a single instrument (\imsim).  Or they were
found to use a  non-supported language (IDL in the case of \laopack).

However, given the relative simplicity of the commonly-adopted
physical model described above, it should be feasible to write a
home-grown turbulence code that is a better match to \galsim\
requirements.   Much of the computation would proceed using
similar DFT techniques to those used in the \texttt{galsim/optics.py} and \texttt{galsim/atmopshere.py}
modules already.   There is freedom in the
construction of a multiple phase screen atmosphere model, including
choice of power spectra, number of screens, screen altitudes, wind
velocities etc., and so some thought will need to go into the user
interface to make this powerful and while being as clean as possible.
 
It would be interesting to learn more about the new method developed
by LSST, and so J\"{o}rg and Barney would welcome any additional
information from those more closely connected with the LSST team, or some
more thorough documentation of the photon shooting method adopted.  It
may be possible to implement both approaches, DFT and photon shooting,
mirroring the twin implementations of object rendering in \galsim.

\end{document}
